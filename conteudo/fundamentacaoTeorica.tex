Métrica de software é qualquer tipo de medida que diz respeito a um sistema de software, processo ou a sua documentação. A principal razão para a medição de um projeto de software é obter informações sobre ele e sobre a organização, e ser capaz de controlar os projetos melhor.

Há muitas mais razões específicas para medir e elas diferem entre perspectiva dos gerentes e do desenvolvedor. Os gestores estão preocupados com questões como: ``qual é o custo do processo?'', ``como é a produtividade da equipe?'', ``o quão bom é o código?'', ``é o cliente satisfeito?'', e ``como podemos fazer melhor?''. Os desenvolvedores se preocupam mais com: ``existem muitas falhas?'', ``podemos testar os requisitos?'', ``temos conseguido atingir os nossos processos e produtos?'', ``o que vai acontecer no futuro?''. A medição de software pode ajudar a manter os gerentes e desenvolvedores informados sobre as suas preocupações, mas não tem a pretensão de dar quaisquer soluções absolutas \cite{lindstrom2004software}).

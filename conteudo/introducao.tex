\subsection{Contexto}
	A Agência Espacial Brasileira \gls{AEB} é um órgão integrante do Sistema de Administração dos Recursos de Tecnologia da Informação do Ministério do Planejamento, Orçamento e Gestão (SISP/MPOG). Um dos principais objetivos da instituição é a busca por uma administração pública que priorize a melhoria de gestão de recursos, e a qualidade na prestação de serviços ao cidadão. Dessa forma, a realização de um bom planejamento de TI que viabilize e potencialize a melhoria contínua da performance organizacional torna-se imprescindível. Vale salientar a necessidade de manter um alinhamento entre as estratégias da instituição com as da unidade de TI \cite{PDTI}.

	No cenário atual de constantes mudanças, a divisão de TI encontra-se em uma fase de reestruturação. Os serviços de TI foram disciplinados em decorrência de regras estabelecidas pela IN04, e devido à complexidade dessas regras e procedimentos houve a necessidade de estruturar a área de Tecnologia da Informação como área estratégica da organização.

	Dentre as principais atividades desempenhadas pela DINF, estão:

	\begin{itemize}
		\item{Elaboração, manutenção e controle do Portfólio de Projetos;}
		\item{Desenvolvimento e manutenção de sistemas;}
		\item{Gerência de segurança;}
		\item{Gerência de máquinas virtuais e servidores;}
		\item{Elaboração, manutenção e controle da Política de Segurança da Informação e Comunicação.}
	\end{itemize}

\subsection{Formulação do problema}

	Após realizar reuniões com o chefe do setor de desenvolvimento da DINF, a equipe de medição obteve acesso à informações de cunho privado que viabilizaram a análise do contexto de desenvolvimento da AEB.

	Nessa vertente, foi possível constatar que nas gestões anteriores, atividades relacionadas à medição e análise não eram tidas como prioridade pela divisão. Então, a partir disso foi proposto a elaboração e execução de um plano de medição a fim de apresentar dados/estimativas reais nos seguintes âmbitos:

	\begin{itemize}
		\item{Produtividade da equipe de desenvolvimento;}
		\item{custo de manutenção dos sistemas implantados;}
		\item{Satisfação do usuário que faz uso desses sistemas.}
	\end{itemize}

\subsection{Objetivos}

	O objetivo geral da pesquisa consiste em tornar claro e evidente aos membros da AEB a importância de utilizar o recurso da medição, seja para medir o processo, produto ou pessoas. Diversos modelos de maturidade, tais como CMMI E Mps Br, e padrões e certificação ISO incluíram práticas de atividades de mensuração como um requisito para processo de desenvolvimento maduro. Além disso, a medição, se conduzida da maneira adequada, provê a identificação de problemas, tomada de decisões com embasamento em indicadores consistentes, e também a melhora no processo da organização.

	Os objetivos específicos pretendidos são conhecer a produtividade da equipe sob o ponto de vista do desenvolvimento de um software; identificar o custo das hora trabalhada de cada integrante da equipe; estimar e avaliar o custo de manutenção de sistemas; avaliar a qualidade dos sistemas que são produzidos e quão satisfeitos os clientes estão ao usarem; analisar o custo-benefício dos sistemas para a equipe de desenvolvimento.


\subsection{Justificativas}

	Medição de software é uma avaliação quantitativa de qualquer aspecto dos processos e produtos de software, que permite seu melhor entendimento e, com isso, auxilia o planejamento, controle e melhoria do que se produz e de como é produzido \cite{bass1999constructing}.

	Embora a equipe de desenvolvimento tenha um nível de conhecimento técnico elevado, seja independente e disciplinada, ainda não há um método consistente que verifique a relação de quanto um membro produz em um determinado período de tempo.

	\begin{quote}
		Sabe-se que um software de baixa qualidade demanda alto custo de manutenção, e consequentemente, diminui a lucratividade da empresa (Como reduzir o Custo de Manutenção de Software com a Análise de Código, 2014).
	\end{quote}

	A título de exemplificação vale citar o sistema de almoxarifado da AEB, cuja qualidade de código é baixa e seu custo de manutenção é elevado, entretanto esse sistema agrada e atende às exigências do cliente. Nessa perspectiva, é importante conhecer em que circunstâncias há viabilidade econômica para que a instituição mantenha determinados sistemas em operação.

	Questões relacionadas ao valor de recursos financeiros repassados à AEB não são de grande preocupação no contexto atual, porém faz-se necessário compreender a relação de custo-benefício dos sistemas implantados. A satisfação do cliente impacta diretamente nessa relação, e com base nesse fator será possível entender se o cliente responde de maneira positiva ou negativo ao que foi entregue a ele.  Em suma, é muito importante averiguar o feedback do usuário, pois ele é o elemento principal que justifica o porquê de desenvolver de um software.

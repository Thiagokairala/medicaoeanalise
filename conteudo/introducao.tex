\subsection{Contexto}
	A AEB é um órgão integrante do Sistema de Administração dos Recursos de Informação e Informática (SISP). Atualmente a divisão de TI se encontra em reformulação, portanto, ele não apresentam nenhum tipo de medição no decorrer do desenvolvimento do projeto.

	Analisando o cenário atual da empresa, nota-se que umas das medições necessárias para avaliar o desempenho das atividades do projeto seriam: produtividade, custo do projeto (desenvolvimento e manutenção), usabilidade e satisfação do usuário.

\subsection{Formulação do problema}
	Atualmente a AEB não executa um processo de medição eficaz, causando diversos problemas de prazo por exemplo. Mediante a isso foi realizada uma reunião com o chefe do setor de desenvolvimento da DINF, e para ele as medições que mais interessavam seriam: produtividade, custo e usabilidade.

	No contexto de produtividade, o cliente gostaria de saber o quanto sua equipe produz em uma quantidade ‘x’ de tempo.

	Como a AEB é uma agência do governo, a preocupação com o custo dos projetos desenvolvidos não é de grande preocupação, tendo em vista que o gasto é fixo, independente de estar ou não em desenvolvimento, porém seria de grande importância  saber o valor gasto com cada projeto, para assim poder estimar os próximos, evitando assim atrasos, que em média, antes da reformulação, eram extremamente frequentes. Além do custo de desenvolvimento, é necessário, segundo o cliente, que o custo de manutenção seja medido, para poder basear análises de troca de produto.

	No que tange à usabilidade, o cliente acredita que seria interessante o levantamento do conforto dos usuários ao utilizar a aplicação.

\subsection{Objetivos}

	O objetivo geral que se pretende atingir durante a execução do trabalho é sensibilizar a AEB sobre a importância da medição, mostrando como o processo auxilia na identificação dos pontos falhos que estão presentes na organização assim como os pontos positivos, e utilizando essas métricas para ajudar na tomada de decisões para melhorar o processo de desenvolvimento de software.

	Os objetivos específicos que pretende-se alcançar com o trabalho são conhecer a produtividade da equipe durante o desenvolvimento de um software, identificar o custo das horas trabalhadas dos integrantes, descobrir o custo de desenvolvimento e de manutenção, analisar a usabilidade dos sistemas que são produzidos, analisar o custo benefício dos sistemas para a equipe de desenvolvimento.

\subsection{Justificativas}

	Medição de software é uma avaliação quantitativa de qualquer aspecto dos processos e produtos de software, que permite seu melhor entendimento e, com isso, auxilia o planejamento, controle e melhoria do que se produz e de como é produzido [BASS et al., 1999].

	As métricas atingidas com o processo de medição proposto podem ter infinitas utilidades como por exemplo:

	\begin{itemize}
		\item{Análise do rendimento dos membros da equipe;}
		\item{Descoberta de desníveis de conhecimento na equipe;}
		\item{Análise crítica sobre a necessidade de descontinuar um sistema;}
		\item{Análise sobre UI e UX dos sistemas desenvolvidos;}
		\item{MAIs COISAS AQUI}
	\end{itemize}

Para o desenvolvimento e implantação do processo de medição, inicialmente será feito o processo de conscientização da equipe de desenvolvimento para que o processo seja entendido não como uma obrigação, e sim como uma agregação de valor ao processo de desenvolvimento e produto.

Após esse processo de conscientização iniciaremos o processo de recolhimento de métricas, aplicando questionários, realizando contagem de horas das atividades, assim como a atribuição de estimativas para cada umas tarefas a serem realizadas.
Por pedido do cliente, não será introduzida uma nova ferramenta de desenvolvimento, sendo apenas utilizada a ferramenta gitlab já preparada e ambientada nos servidores da AEB.

As medidas de tempo e produtividade então serão marcadas dentro dos commits dados pelos desenvolvedores, apenas colocando o padrão TIME XX:XX onde o XX:XX será a duração da tarefa.

Finalmente após obter as métricas será possível analisá-las, a fim de descobrir se o processo e o produto estão eficazes e se estão sendo aplicados da melhor forma possível, e caso necessário aplicar melhorias, e recomeçar o processo.

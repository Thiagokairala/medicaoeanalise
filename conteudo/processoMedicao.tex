\subsection{Medições}

\subsubsection{Produtividade}

\begin{table}[H]
\centering
\begin{tabular}{|p{4cm}|p{7cm}|}
\hline
	%-------------------------------------------------------
	\textbf{Objetivo da medição} &
	Esforço gasto para concluir cada issue.
	\\ \hline
	%-------------------------------------------------------
	\textbf{Fórmula} &
	$E = \dfrac{Horas trabalhadas}{Estimativa de horas de cada issue}$
	\\ \hline
	%-------------------------------------------------------
	\textbf{Escala da medição} &
	Racional
	\\ \hline
	%-------------------------------------------------------
	\textbf{Coleta} &
	\begin{itemize}
		\item{Responsável: Thiago Kairala}
		\item{Periodicidade ou Evento: Semanalmente.}
		\item{Procedimentos: Inclusão no commit de fechamento a frase TIME XX:XX onde os X's representam o tempo}
	\end{itemize}
	\\ \hline
	%-------------------------------------------------------
	\textbf{Análise} &
	\begin{itemize}
		\item Responsável: Indiara Duarte
		\item Procedimentos: Identificar possíveis causas de desvios e possíveis ações.
	\end{itemize}
	\\ \hline
	%-------------------------------------------------------
	\textbf{Meta} &
	Limites de especificação
		\begin{itemize}
			\item Inferior: 0,9
			\item Superior: 1,1
		\end{itemize}
	Limites de controle
		\begin{itemize}
			\item Inferior: 0,95
			\item Superior: 1,05
		\end{itemize}
  \\ \hline
\end{tabular}
\caption{Efetividade da estimativa}
\label{tab:efetividade_estimativa}
\end{table}

\subsubsection{Custo de manutenção}


\begin{table}[H]
\centering
\begin{tabular}{|p{4cm}|p{7cm}|}
\hline
	%-------------------------------------------------------
	\textbf{Objetivo da medição} &
	Custo
	\\ \hline
	%-------------------------------------------------------
	\textbf{Fórmula} &
	$C = Horas trabalhadas * Estimativa de horas de cada issue$
	\\ \hline
	%-------------------------------------------------------
	\textbf{Escala da medição} &
	Racional
	\\ \hline
	%-------------------------------------------------------
	\textbf{Coleta} &
	\begin{itemize}
		\item{Responsável: Indiara Duarte}
		\item{Periodicidade ou Evento: Quinzenalmente.}
		\item{Procedimentos:  Inclusão no commit de fechamento a frase TIME XX:XX onde os X's representam o tempo}
	\end{itemize}
	\\ \hline
	%-------------------------------------------------------
	\textbf{Análise} &
	\begin{itemize}
		\item Responsável: Thiago Kairala
		\item Procedimentos: Identificar possíveis causas de desvios e possíveis ações.
	\end{itemize}
	\\ \hline
	%-------------------------------------------------------
	\textbf{Meta} &
	Limites de especificação
		Os limites devem variar de projeto para projeto.
  \\ \hline
\end{tabular}
\caption{Efetividade da estimativa}
\label{tab:efetividade_estimativa}
\end{table}

\subsubsection{Usabilidade}

\begin{table}[H]
\centering
\begin{tabular}{|p{4cm}|p{5cm}|}
\hline
	%-------------------------------------------------------
	\textbf{Objetivo da medição} &
	Analisar a qualiidade da simplicidade do sistema
	\\ \hline
	%-------------------------------------------------------
	\textbf{Fórmula} &
	$\dfrac{Total de pontos}{número de questionários}$
	\\ \hline
	%-------------------------------------------------------
	\textbf{Escala da medição} &
	Ordinal
	\\ \hline
	%-------------------------------------------------------
	\textbf{Coleta} &
	\begin{itemize}
		\item Responsável: \stefania
		\item Procedimentos: Realizar o questionário
	\end{itemize}
	\\ \hline
	%-------------------------------------------------------
	\textbf{Análise} &
	\begin{itemize}
		\item Responsável: \fabio
		\item Procedimentos: Identificar possíveis causas de desvios e possíveis ações.
	\end{itemize}
	\\ \hline
  %-------------------------------------------------------
  \textbf{Meta} &
	Limites de especificação
		\begin{itemize}
			\item Inferior: 3,5
			\item Superior: N/A
		\end{itemize}
	Limites de controle
		\begin{itemize}
			\item Inferior: 4,0
			\item Superior: N/A
		\end{itemize}
  \\ \hline
\end{tabular}
\caption{Satisfação com simplicidade}
\label{tab:satisfacao_simplicidade}
\end{table}

\begin{table}[H]
\centering
\begin{tabular}{|p{4cm}|p{5cm}|}
\hline
	%-------------------------------------------------------
	\textbf{Objetivo da medição} &
	Analisar se o sistema atende realmente as necessidades do cliente
	\\ \hline
	%-------------------------------------------------------
	\textbf{Fórmula} &
	$\dfrac{Total de pontos}{número de questionários}$
	\\ \hline
	%-------------------------------------------------------
	\textbf{Escala da medição} &
	Ordinal
	\\ \hline
	%-------------------------------------------------------
	%-------------------------------------------------------
	\textbf{Coleta} &
	\begin{itemize}
		\item Responsável: \fabio
		\item Procedimentos: Realizar o questionário
	\end{itemize}
	\\ \hline
	%-------------------------------------------------------
	\textbf{Análise} &
	\begin{itemize}
		\item Responsável: \stefania
		\item Procedimentos: Identificar possíveis causas de desvios e possíveis ações.
	\end{itemize}
	\\ \hline
  %-------------------------------------------------------
  \textbf{Meta} &
	Limites de especificação
		\begin{itemize}
			\item Inferior: 4,0
			\item Superior: N/A
		\end{itemize}
	Limites de controle
		\begin{itemize}
			\item Inferior: 4,5
			\item Superior: N/A
		\end{itemize}
  \\ \hline
\end{tabular}
\caption{Satisfação com complitude}
\label{tab:satisfacao_complitude}
\end{table}

\begin{table}[H]
\centering
\begin{tabular}{|p{4cm}|p{5cm}|}
\hline
	%-------------------------------------------------------
	\textbf{Objetivo da medição} &
	Analisar o tempo de atendimento na visão do cliente
	\\ \hline
	%-------------------------------------------------------
	\textbf{Fórmula} &
	$\dfrac{Total de pontos}{número de questionários}$
	\\ \hline
	%-------------------------------------------------------
	\textbf{Escala da medição} &
	Ordinal
	\\ \hline
	%-------------------------------------------------------
	%-------------------------------------------------------
	\textbf{Coleta} &
	\begin{itemize}
		\item Responsável: \stefania
		\item Procedimentos: Realizar o questionário
	\end{itemize}
	\\ \hline
	%-------------------------------------------------------
	\textbf{Análise} &
	\begin{itemize}
		\item Responsável: \fabio
		\item Procedimentos: Identificar possíveis causas de desvios e possíveis ações.
	\end{itemize}
	\\ \hline
	%-------------------------------------------------------
  \textbf{Meta} &
	Limites de especificação
		\begin{itemize}
			\item Inferior: 3,5
			\item Superior: N/A
		\end{itemize}
	Limites de controle
		\begin{itemize}
			\item Inferior: 4,0
			\item Superior: N/A
		\end{itemize}
  \\ \hline
\end{tabular}
\caption{Satisfação com a velocidade de atendimento}
\label{tab:satisfacao_velocidade}
\end{table}

\subsection{Abstraction Sheet}

\subsubsection{Produtividade da equipe}

\begin{table}[H]
\centering
\begin{tabular}{|p{4cm}|p{4cm}|}
\hline
	%-------------------------------------------------------
	\begin{center}
	\textbf{Foco de qualidade}
	\end{center}

	\begin{itemize}
		\item Esforço
	\end{itemize}

	&

	%-------------------------------------------------------
	\begin{center}
	\textbf{Fatores de variação}
	\end{center}

	\begin{itemize}
		\item{Tamanho da equipe;}
		\item{Motivação da equipe;}
		\item{Desnivelamento de conhecimento;}
	\end{itemize}

	\\ \hline
	%-------------------------------------------------------
	\begin{center}
	\textbf{Hipótese de baseline}
	\end{center}

	&

	\begin{center}
	\textbf{Hipótese de baseline}
	\end{center}

	\\ \hline
\end{tabular}
\caption{Abstraction sheet - Produtivade}
\label{tab:produtividade_sheet}
\end{table}

\subsubsection{Custo de manutenção}

\begin{table}[H]
\centering
\begin{tabular}{|p{4cm}|p{4cm}|}
\hline
	%-------------------------------------------------------
	\begin{center}
	\textbf{Foco de qualidade}
	\end{center}

	\begin{itemize}
		\item Custo
	\end{itemize}

	&

	%-------------------------------------------------------
	\begin{center}
	\textbf{Fatores de variação}
	\end{center}

	\begin{itemize}
		\item{Salário do executor}
		\item{Tempo para conclusão da issue;}
		\item{Criticidade de falha.}
	\end{itemize}

	\\ \hline
	%-------------------------------------------------------
	\begin{center}
	\textbf{Hipótese de baseline}
	\end{center}

	&

	\begin{center}
	\textbf{Hipótese de baseline}
	\end{center}

	\\ \hline
\end{tabular}
\caption{Abstraction sheet - Custo de manutenção}
\label{tab:manutencao_sheet}
\end{table}

\subsubsection{Satisfação do cliente}

\begin{table}[H]
\centering
\begin{tabular}{|p{4cm}|p{4cm}|}
\hline
	%-------------------------------------------------------
	\begin{center}
	\textbf{Foco de qualidade}
	\end{center}

	\begin{itemize}
		\item{Eficiência;}
		\item{Usabilidade;}
		\item{Prazo de entrega;}
		\item{Confiabilidade.}
	\end{itemize}

	&

	%-------------------------------------------------------
	\begin{center}
	\textbf{Fatores de variação}
	\end{center}

	\begin{itemize}
		\item{Dificuldade no uso do sistema;}
		\item{Falhas constantes;}
		\item{Atraso na entrega do sistema;}
	\end{itemize}

	\\ \hline
	%-------------------------------------------------------
	\begin{center}
	\textbf{Hipótese de baseline}
	\end{center}

	&

	\begin{center}
	\textbf{Hipótese de baseline}
	\end{center}

	\\ \hline
\end{tabular}
\caption{Abstraction sheet - Produtivade}
\label{tab:produtividade_sheet}
\end{table}

\subsection{GQM}


\begin{table}[H]
\centering
\begin{tabular}{|c|c|c|}
\hline
	%-------------------------------------------------------
	\textbf{Goal} &
  \textbf{Question} &
  \textbf{Metrics}
	\\ \hline
	%-------------------------------------------------------
	 &
   &

	\\ \hline
\end{tabular}
\caption{Métrica de usabilidade}
\label{tab:métrica_de_usabilidade}
\end{table}

\subsection{Modelo de maturidade}
	A escolha do modelo de maturidade atuante no projeto foi o MPS-BR.

  A escolha do modelo de maturidade atuante no projeto foi o MPS-BR. Tendo em vista que esse modelo é o que mais se adapta a realidade brasileira, tendo em vista que para se obter um certificado do CMMI  tem custo entre duzentos mil reais à um milhão de reais.

  O propósito do processo Medição é coletar, armazenar, analisar e relatar os dados relativos aos produtos desenvolvidos e aos processos implementados na organização e em seus projetos, de forma a apoiar os objetivos organizacionais (SOFTEX, 2012) !!!!! fazer citaçao no latex.

  \begin{table}[H]
  \centering
  \begin{tabular}{|p{3cm}|p{5cm}|p{3cm}|}
  \hline
    \multicolumn{3}{|c|}{\textbf{Medição - MED}} \\ \hline
    \textbf{Identificador} & \textbf{Resultado esperado} & \textbf{Aplicação no projeto} \\ \hline
    \textbf{MED 1} & Objetivos de medição são estabelecidos e mantidos a partir dos objetivos de negócio da organização e das necessidades de informação de processos técnicos e gerenciais; & Essa etapa é realizada quando há entrevista com o diretor da organização. \\ \hline
    \textbf{MED 2} & Um conjunto adequado de medidas, orientado pelos objetivos de medição, é identificado e definido, priorizado, documentado, revisados e, quando pertinente, atualizado; & Essa etapa é realizada quando se faz a (atividade que esqueci o nome). \\ \hline
    \textbf{MED 3} & Os procedimentos para a coleta e o armazenamento de medidas são especificados; & Essa etapa é realizada quando se coleta os dados para observar os resultados obtidos e fazer comparações quando necessário. \\ \hline
    \textbf{MED 4} & Os procedimentos para a análise das medidas são especificados; & Essa etapa é realizada quando se coleta as métricas e analisadas de acordo com o GQM. \\ \hline
    \textbf{MED 5} & Os dados requeridos são coletados e analisados; & Essa etapa é feita para que haja um interpretação dos dados. \\ \hline
    \textbf{MED 6} & Os dados e os resultados das análises são armazenados; & Os dados são armazenados para que haja comparação no decorrer do tempo em que se está analisando a organização. \\ \hline
    \textbf{MED 7} & Os dados e os resultados das análises são comunicados aos interessados e são utilizados para apoiar decisões. & Ao final do projeto, os dados serão disponibilizados a organização, informando como o processo pode ser mudado. \\ \hline
  \end{tabular}
  \caption{Medições????? ARRUMAR ESSA LEGENDA}
  \label{tab:medicoes}
  \end{table}

\subsection{Medições}

\subsubsection{Produtividade}

\begin{table}[H]
\centering
\begin{tabular}{|p{4cm}|p{5cm}|}
\hline
	%-------------------------------------------------------
	\textbf{Objetivo da medição} &

	\\ \hline
	%-------------------------------------------------------
	\textbf{Fórmula} &

	\\ \hline
	%-------------------------------------------------------
	\textbf{Escala da medição} &

	\\ \hline
	%-------------------------------------------------------
	\textbf{Coleta} &

	\\ \hline
	%-------------------------------------------------------
	\textbf{Responsável} &

	\\ \hline
	%-------------------------------------------------------
	\textbf{Periodicidade ou evento} &

	\\ \hline
  %-------------------------------------------------------
	\textbf{Procedimento} &

	\\ \hline
  %-------------------------------------------------------
  \textbf{Meta} &

  \\ \hline
\end{tabular}
\caption{Métrica de produtividade}
\label{tab:métrica_de_produtividade}
\end{table}


\subsubsection{Custo}

\begin{table}[H]
\centering
\begin{tabular}{|p{4cm}|p{5cm}|}
\hline
	%-------------------------------------------------------
	\textbf{Objetivo da medição} &

	\\ \hline
	%-------------------------------------------------------
	\textbf{Fórmula} &

	\\ \hline
	%-------------------------------------------------------
	\textbf{Escala da medição} &

	\\ \hline
	%-------------------------------------------------------
	\textbf{Coleta} &

	\\ \hline
	%-------------------------------------------------------
	\textbf{Responsável} &

	\\ \hline
	%-------------------------------------------------------
	\textbf{Periodicidade ou evento} &

	\\ \hline
  %-------------------------------------------------------
	\textbf{Procedimento} &

	\\ \hline
  %-------------------------------------------------------
  \textbf{Meta} &

  \\ \hline
\end{tabular}
\caption{Métrica de custo}
\label{tab:métrica_de_custo}
\end{table}


\subsubsection{Usabilidade}

\begin{table}[H]
\centering
\begin{tabular}{|p{4cm}|p{5cm}|}
\hline
	%-------------------------------------------------------
	\textbf{Objetivo da medição} &

	\\ \hline
	%-------------------------------------------------------
	\textbf{Fórmula} &

	\\ \hline
	%-------------------------------------------------------
	\textbf{Escala da medição} &

	\\ \hline
	%-------------------------------------------------------
	\textbf{Coleta} &

	\\ \hline
	%-------------------------------------------------------
	\textbf{Responsável} &

	\\ \hline
	%-------------------------------------------------------
	\textbf{Periodicidade ou evento} &

	\\ \hline
  %-------------------------------------------------------
	\textbf{Procedimento} &

	\\ \hline
  %-------------------------------------------------------
  \textbf{Meta} &

  \\ \hline
\end{tabular}
\caption{Métrica de usabilidade}
\label{tab:métrica_de_usabilidade}
\end{table}

\subsection{Abstraction Sheet}

\subsection{GQM}


\begin{table}[H]
\centering
\begin{tabular}{|c|c|c|}
\hline
	%-------------------------------------------------------
	\textbf{Goal} &
  \textbf{Question} &
  \textbf{Metrics}
	\\ \hline
	%-------------------------------------------------------
	 &
   &

	\\ \hline
\end{tabular}
\caption{Métrica de usabilidade}
\label{tab:métrica_de_usabilidade}
\end{table}

\subsection{Modelo de maturidade}

  A escolha do modelo de maturidade atuante no projeto foi o MPS-BR. Tendo em vista que esse modelo é o que mais se adapta a realidade brasileira, tendo em vista que para se obter um certificado do CMMI  tem custo entre duzentos mil reais à um milhão de reais.

  O propósito do processo Medição é coletar, armazenar, analisar e relatar os dados relativos aos produtos desenvolvidos e aos processos implementados na organização e em seus projetos, de forma a apoiar os objetivos organizacionais (SOFTEX, 2012) !!!!! fazer citaçao no latex.

  \begin{table}[]
  \centering
  \begin{tabular}{|l|l|l|}
  \hline
    \multicolumn{3}{|c|}{\textbf{Medição - MED}} \\ \hline
    \textbf{Identificador} & \textbf{Resultado esperado} & \textbf{Aplicação no projeto} \\ \hline
    \textbf{MED 1} & Objetivos de medição são estabelecidos e mantidos a partir dos objetivos de negócio da organização e das necessidades de informação de processos técnicos e gerenciais; & Essa etapa é realizada quando há entrevista com o diretor da organização. \\ \hline
    \textbf{MED 2} & Um conjunto adequado de medidas, orientado pelos objetivos de medição, é identificado e definido, priorizado, documentado, revisados e, quando pertinente, atualizado; & Essa etapa é realizada quando se faz a (atividade que esqueci o nome). \\ \hline
    \textbf{MED 3} & Os procedimentos para a coleta e o armazenamento de medidas são especificados; & Essa etapa é realizada quando se coleta os dados para observar os resultados obtidos e fazer comparações quando necessário. \\ \hline
    \textbf{MED 4} & Os procedimentos para a análise das medidas são especificados; & Essa etapa é realizada quando se coleta as métricas e analisadas de acordo com o GQM. \\ \hline
    \textbf{MED 5} & Os dados requeridos são coletados e analisados; & Essa etapa é feita para que haja um interpretação dos dados. \\ \hline
    \textbf{MED 6} & Os dados e os resultados das análises são armazenados; & Os dados são armazenados para que haja comparação no decorrer do tempo em que se está analisando a organização. \\ \hline
    \textbf{MED 7} & Os dados e os resultados das análises são comunicados aos interessados e são utilizados para apoiar decisões. & Ao final do projeto, os dados serão disponibilizados a organização, informando como o processo pode ser mudado. \\ \hline
  \end{tabular}
  \caption{Medições????? ARRUMAR ESSA LEGENDA}
  \label{tab:medicoes}
  \end{table}

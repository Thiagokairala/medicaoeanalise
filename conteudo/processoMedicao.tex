\subsection{Medições}

\subsubsection{Produtividade}

\begin{table}[H]
\centering
\begin{tabular}{|p{4cm}|p{5cm}|}
\hline
	%-------------------------------------------------------
	\textbf{Objetivo da medição} &
	Medir o tempo de realização de cada tarefa
	\\ \hline
	%-------------------------------------------------------
	\textbf{Fórmula} &
	$Tempo Inicial - Tempo final$
	\\ \hline
	%-------------------------------------------------------
	\textbf{Escala da medição} &

	\\ \hline
	%-------------------------------------------------------
	\textbf{Coleta} &
	A coleta é feita inserindo a mensagem \#<numero da issue> XX:XX onde os x representam o tempo gasto, e será feita por todos os desenvolvedores.
	\\ \hline
	%-------------------------------------------------------
	\textbf{Análise} &
	A análise será realizada no fechamento de cada sprint, pelo chefe de desenvolvimento, para poder analizar com precisão o tempo gasto pelos desenvolvedores.
	\\ \hline
	%-------------------------------------------------------
	\textbf{Procedimento} &
	Os possíveis desvios são tempos sub declarados ou super declarados a fim de parecer que a tarefa era mais fácil ou mais difícil do que realmente era.
	\\ \hline
  %-------------------------------------------------------
  \textbf{Meta} &
	Todas as tarefas devem ser feitas entre 4 e 40 horas.
  \\ \hline
\end{tabular}
\caption{Tempo de realização de tarefa}
\label{tab:tempo_realizacao_tarefa}
\end{table}


\begin{table}[H]
\centering
\begin{tabular}{|p{4cm}|p{5cm}|}
\hline
	%-------------------------------------------------------
	\textbf{Objetivo da medição} &
	Estimar quanto tempo uma tarefa deve durar
	\\ \hline
	%-------------------------------------------------------
	\textbf{Fórmula} &
	O cálculo é feito por estimativa de quantas horas cada tarefa deve durar.
	\\ \hline
	%-------------------------------------------------------
	\textbf{Escala da medição} &

	\\ \hline
	%-------------------------------------------------------
	\textbf{Coleta} &
	A coleta é feita no momento de priorização da tarefa, colocando na descrição da mesma, pelo chefe de desenvolvimento.
	\\ \hline
	%-------------------------------------------------------
	\textbf{Análise} &
	A análise é feita no planejamento da sprint, para garantir que não seja proposto mais do que a equipe tenha capacidade de entregar.
	\\ \hline
	%-------------------------------------------------------
	\textbf{Procedimento} &
	Os possíveis desvios são subestimação ou superestimação da tarefa, causando uma estimativa muito maior ou menor que o real, podendo causar atraso ou dias sem atividade.
	\\ \hline
  %-------------------------------------------------------
  \textbf{Meta} &
	Todas as tarefas devem ser feitas entre 4 a 40 horas.
  \\ \hline
\end{tabular}
\caption{Tempo estimado para realização da tarefa}
\label{tab:tempo_estimado_realização_tarefa}
\end{table}

\subsubsection{Custo}


\begin{table}[H]
\centering
\begin{tabular}{|p{4cm}|p{5cm}|}
\hline
	%-------------------------------------------------------
	\textbf{Objetivo da medição} &
	Analisar o custo de desenvolvimento de uma tarefa
	\\ \hline
	%-------------------------------------------------------
	\textbf{Fórmula} &
	$Tempo de realização * Custo da hora do executor$
	\\ \hline
	%-------------------------------------------------------
	\textbf{Escala da medição} &

	\\ \hline
	%-------------------------------------------------------
	\textbf{Coleta} &
	A coleta é feita no momento de fechamento da sprint pelo Chefe de desenvolvimento.
	\\ \hline
	%-------------------------------------------------------
	\textbf{Análise} &
	A análise será feita ao final do projeto, para se analizar o custo final, e poder apresentar para a alta direção da AEB.
	\\ \hline
	%-------------------------------------------------------
	\textbf{Procedimento} &
	Os possíveis desvios são causados pelo possível desvio na medição de tempo, podendo aumentar ou diminuir o custo final do projeto.
	\\ \hline
  %-------------------------------------------------------
  \textbf{Meta} &
	A meta depende de projeto para projeto.
  \\ \hline
\end{tabular}
\caption{Custo por tarefa}
\label{tab:custo_por_tarefa}
\end{table}

\begin{table}[H]
\centering
\begin{tabular}{|p{4cm}|p{5cm}|}
\hline
	%-------------------------------------------------------
	\textbf{Objetivo da medição} &
	Analisar o custo de manuteção de cada sistema.
	\\ \hline
	%-------------------------------------------------------
	\textbf{Fórmula} &
	$Tempo de realização * Custo da hora do executor$
	\\ \hline
	%-------------------------------------------------------
	\textbf{Escala da medição} &

	\\ \hline
	%-------------------------------------------------------
	\textbf{Coleta} &
	A coleta é feita a cada vez que uma tarefa de manutenção for executada.
	\\ \hline
	%-------------------------------------------------------
	\textbf{Análise} &
	A análise será feita quando exixtir a necessidade de análise sobre a continuidade ou não de um projeto.
	\\ \hline
	%-------------------------------------------------------
	\textbf{Procedimento} &
	Os possíveis desvios são causados pelo possível desvio na medição de tempo, podendo aumentar ou diminuir o custo final da manutenção.
	\\ \hline
  %-------------------------------------------------------
  \textbf{Meta} &
	A meta depende de projeto para projeto.
  \\ \hline
\end{tabular}
\caption{Custo de manutenção}
\label{tab:custo_de_manutenção}
\end{table}

\subsubsection{Usabilidade}

\begin{table}[H]
\centering
\begin{tabular}{|p{4cm}|p{5cm}|}
\hline
	%-------------------------------------------------------
	\textbf{Objetivo da medição} &
	Analisar a qualiidade da simplicidade do sistema
	\\ \hline
	%-------------------------------------------------------
	\textbf{Fórmula} &
	$\dfrac{Total de pontos}{número de questionários}$
	\\ \hline
	%-------------------------------------------------------
	\textbf{Escala da medição} &

	\\ \hline
	%-------------------------------------------------------
	\textbf{Coleta} &
	A coleta é feita sempre que uma funcionalidade grande for entregue, por algum integrante da equipe de desenvolvimento, realizando um questionário para os clientes que utilizam cada sistema.
	\\ \hline
	%-------------------------------------------------------
	\textbf{Análise} &
	A análise será feita sempre que for coletada a métrica.
	\\ \hline
	%-------------------------------------------------------
	\textbf{Procedimento} &
	Existem possíveis desvios devido ao fato da métrica ser subjetiva.
	\\ \hline
  %-------------------------------------------------------
  \textbf{Meta} &
	A meta da métrica é estar sempre a cima de 3.5, numa escala de 1 a 5.
  \\ \hline
\end{tabular}
\caption{Satisfação com simplicidade}
\label{tab:satisfacao_simplicidade}
\end{table}

\begin{table}[H]
\centering
\begin{tabular}{|p{4cm}|p{5cm}|}
\hline
	%-------------------------------------------------------
	\textbf{Objetivo da medição} &
	Analisar se o sistema atende realmente as necessidades do cliente
	\\ \hline
	%-------------------------------------------------------
	\textbf{Fórmula} &
	$\dfrac{Total de pontos}{número de questionários}$
	\\ \hline
	%-------------------------------------------------------
	\textbf{Escala da medição} &

	\\ \hline
	%-------------------------------------------------------
	\textbf{Coleta} &
	A coleta é feita sempre que uma funcionalidade grande for entregue, por algum integrante da equipe de desenvolvimento, realizando um questionário para os clientes que utilizam cada sistema.
	\\ \hline
	%-------------------------------------------------------
	\textbf{Análise} &
	A análise será feita sempre que for coletada a métrica.
	\\ \hline
	%-------------------------------------------------------
	\textbf{Procedimento} &
	Existem possíveis desvios devido ao fato da métrica ser subjetiva.
	\\ \hline
  %-------------------------------------------------------
  \textbf{Meta} &
	A meta da métrica é estar sempre a cima de 4.5, numa escala de 1 a 5.
  \\ \hline
\end{tabular}
\caption{Satisfação com complitude}
\label{tab:satisfacao_complitude}
\end{table}

\begin{table}[H]
\centering
\begin{tabular}{|p{4cm}|p{5cm}|}
\hline
	%-------------------------------------------------------
	\textbf{Objetivo da medição} &
	Analisar o tempo de atendimento na visão do cliente
	\\ \hline
	%-------------------------------------------------------
	\textbf{Fórmula} &
	$\dfrac{Total de pontos}{número de questionários}$
	\\ \hline
	%-------------------------------------------------------
	\textbf{Escala da medição} &

	\\ \hline
	%-------------------------------------------------------
	\textbf{Coleta} &
	A coleta é feita sempre que uma funcionalidade grande for entregue, por algum integrante da equipe de desenvolvimento, realizando um questionário para os clientes que utilizam cada sistema.
	\\ \hline
	%-------------------------------------------------------
	\textbf{Análise} &
	A análise será feita sempre que for coletada a métrica.
	\\ \hline
	%-------------------------------------------------------
	\textbf{Procedimento} &
	Existem possíveis desvios devido ao fato da métrica ser subjetiva.
	\\ \hline
  %-------------------------------------------------------
  \textbf{Meta} &
	A meta da métrica é estar sempre a cima de 3.5, numa escala de 1 a 5.
  \\ \hline
\end{tabular}
\caption{Satisfação com a velocidade de atendimento}
\label{tab:satisfacao_velocidade}
\end{table}

\subsection{Abstraction Sheet}

\subsection{GQM}


\begin{table}[H]
\centering
\begin{tabular}{|c|c|c|}
\hline
	%-------------------------------------------------------
	\textbf{Goal} &
  \textbf{Question} &
  \textbf{Metrics}
	\\ \hline
	%-------------------------------------------------------
	 &
   &

	\\ \hline
\end{tabular}
\caption{Métrica de usabilidade}
\label{tab:métrica_de_usabilidade}
\end{table}

\subsection{Modelo de maturidade}

  A escolha do modelo de maturidade atuante no projeto foi o MPS-BR. Tendo em vista que esse modelo é o que mais se adapta a realidade brasileira, tendo em vista que para se obter um certificado do CMMI  tem custo entre duzentos mil reais à um milhão de reais.

  O propósito do processo Medição é coletar, armazenar, analisar e relatar os dados relativos aos produtos desenvolvidos e aos processos implementados na organização e em seus projetos, de forma a apoiar os objetivos organizacionais (SOFTEX, 2012) !!!!! fazer citaçao no latex.

  \begin{table}[H]
  \centering
  \begin{tabular}{|p{3cm}|p{5cm}|p{3cm}|}
  \hline
    \multicolumn{3}{|c|}{\textbf{Medição - MED}} \\ \hline
    \textbf{Identificador} & \textbf{Resultado esperado} & \textbf{Aplicação no projeto} \\ \hline
    \textbf{MED 1} & Objetivos de medição são estabelecidos e mantidos a partir dos objetivos de negócio da organização e das necessidades de informação de processos técnicos e gerenciais; & Essa etapa é realizada quando há entrevista com o diretor da organização. \\ \hline
    \textbf{MED 2} & Um conjunto adequado de medidas, orientado pelos objetivos de medição, é identificado e definido, priorizado, documentado, revisados e, quando pertinente, atualizado; & Essa etapa é realizada quando se faz a (atividade que esqueci o nome). \\ \hline
    \textbf{MED 3} & Os procedimentos para a coleta e o armazenamento de medidas são especificados; & Essa etapa é realizada quando se coleta os dados para observar os resultados obtidos e fazer comparações quando necessário. \\ \hline
    \textbf{MED 4} & Os procedimentos para a análise das medidas são especificados; & Essa etapa é realizada quando se coleta as métricas e analisadas de acordo com o GQM. \\ \hline
    \textbf{MED 5} & Os dados requeridos são coletados e analisados; & Essa etapa é feita para que haja um interpretação dos dados. \\ \hline
    \textbf{MED 6} & Os dados e os resultados das análises são armazenados; & Os dados são armazenados para que haja comparação no decorrer do tempo em que se está analisando a organização. \\ \hline
    \textbf{MED 7} & Os dados e os resultados das análises são comunicados aos interessados e são utilizados para apoiar decisões. & Ao final do projeto, os dados serão disponibilizados a organização, informando como o processo pode ser mudado. \\ \hline
  \end{tabular}
  \caption{Medições????? ARRUMAR ESSA LEGENDA}
  \label{tab:medicoes}
  \end{table}

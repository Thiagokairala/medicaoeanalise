\subsection{Resumo da proposta}
	A proposta que será abordada pelo grupo é de coletar métricas referentes a custo, produtividade e usabilidade, melhorando a qualidade de desenvolvimento de software, com o propósito de agregar mais valor para a empresa.

\subsection{Lista de Software}
  Os softwares utilizados para o projeto serão apenas os softwares já utilizados na AEB, como o gitlab, o GNU plan além de um editor de documentos para a criação dos questionários de satisfação.

\subsection{Descrição de atividades}

\begin{table}[H]
\centering
\begin{tabular}{|p{1cm}|p{2cm}|p{5cm}|p{3cm}|}
\hline
	%-------------------------------------------------------
	\textbf{Id} &
	\textbf{Atividade} &
	\textbf{Descrição} &
  \textbf{Responsáveis}
	\\ \hline
	%-------------------------------------------------------
	A1 &
	Reunião com o responsável da equipe de desenvolvimento da DNIF &
	Reunião presencial com o responsável da DINF na AEB para compreender o contexto do trabalho realizado pelos desenvolvedores e para a afirmação de um acordo(verbal) de solicitação de dados dos projetos realizados. &
	Time de medição e análise.
	\\ \hline
	%-------------------------------------------------------
	A2 &
	Compreender as necessidades da AEB quanto a medição e análise. &
	Tendo como base a reunião com o responsável da DINF da AEB o time de medição e análise poderá esboçar quais as possíveis aplicações  da medição e análise no DINF. &
	Time de medição e análise.
	\\ \hline

\end{tabular}
\caption{Fase 1. Entendimento do escopo do projeto}
\label{tab:atividades_fase_1}
\end{table}

\begin{table}[H]
\centering
\begin{tabular}{|p{1cm}|p{2cm}|p{5cm}|p{3cm}|}
\hline
	%-------------------------------------------------------
	\textbf{Id} &
	\textbf{Atividade} &
	\textbf{Descrição} &
  \textbf{Responsáveis}
	\\ \hline
	%-------------------------------------------------------
	A3 &
	Definir quais projetos da AEB serão analisados. &
	Com base nas necessidades da AEB quanto a medição e análise, definir quais dos projetos da AEB serão analisados. &
	Time de medição e análise.
	\\ \hline
	%-------------------------------------------------------
	A4 &
	Definir medições a serem realizadas &
	Escolher que tipo de medições serão realizadas nos projetos escolhidos. &

	\\ \hline
	%-------------------------------------------------------
	A5 &
	Definir ferramentas utilizadas na medição. &
	Escolher quais ferramentas serão utilizadas para a coleta e análise dos dados. &

	\\ \hline
	%-------------------------------------------------------
	A6 &
	Elaborar Abstraction Sheet &
	 &

	\\ \hline
	%-------------------------------------------------------
	A7 &
	Elaborar GQM &
	 &

	\\ \hline
\end{tabular}
\caption{Fase 2. Planejar processo de medição e análise}
\label{tab:atividades_fase_2}
\end{table}

\begin{table}[H]
\centering
\begin{tabular}{|p{1cm}|p{2cm}|p{5cm}|p{3cm}|}
\hline
	%-------------------------------------------------------
	\textbf{Id} &
	\textbf{Atividade} &
	\textbf{Descrição} &
  \textbf{Responsáveis}
	\\ \hline
	%-------------------------------------------------------
	A8 &
	Identificar os riscos de cada projeto. &
	Identificar os riscos de cada projeto escolhido para medição e análise. &

	\\ \hline
	%-------------------------------------------------------
	A9 &
	Identificar os custos de cada projeto &
	Identificar os custos de cada projeto escolhido para medição e análise. &

	\\ \hline
	%-------------------------------------------------------
	A10 &
	Identificar a produtividade da equipe da AEB em cada projeto. &
	Identificar a produtividade da equipe de desenvolvimento do DNIF em cada uma dos projetos analisados. &

	\\ \hline
	%-------------------------------------------------------
	A11 &
	Identificar a satisfação do cliente em cada projeto. &
	Identificar a satisfação dos usuários em cada um dos projetos escolhidos para análise. &

	\\ \hline
	%-------------------------------------------------------
	A12 &
	Coletar métricas de código fonte. &
	Com a devida permissão do responsável DINF serão coletadas métricas de código fonte dos projetos escolhidos. &

	\\ \hline
\end{tabular}
\caption{Fase 3. Realizar medições}
\label{tab:atividades_fase_3}
\end{table}

\begin{table}[H]
\centering
\begin{tabular}{|p{1cm}|p{2cm}|p{5cm}|p{3cm}|}
\hline
	%-------------------------------------------------------
	\textbf{Id} &
	\textbf{Atividade} &
	\textbf{Descrição} &
  \textbf{Responsáveis}
	\\ \hline
	%-------------------------------------------------------
	A13 &
	Diagnosticar sobre custos de cada projeto &
	 &

	\\ \hline
	%-------------------------------------------------------
	A14 &
	Diagnosticar sobre  a qualidade do código fonte de cada projeto &
   &

	\\ \hline
	%-------------------------------------------------------
	A15 &
	Diagnosticar sobre a satisfação dos usuários. &
	 &

	\\ \hline
\end{tabular}
\caption{Fase 4. Resultados obtidos}
\label{tab:atividades_fase_4}
\end{table}

\subsection{Cronograma}
  [Inserir aqui um cronograma com as atividades (no mínimo) citadas na seção anterior. Representar a unidade de tempo como pelo menos mensal, podendo ser semanal ou diário]
